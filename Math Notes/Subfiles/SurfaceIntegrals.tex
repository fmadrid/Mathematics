\documentclass[../main.tex]{subfiles}
\begin{document}
The relationship between surface integrals and surface area is similar to the relationship between line integrals and arc length.

\subsection{Parametric Surfaces}
To evaluate the surface integrals, we approximate path area $\Delta S_{ij}$ by the area of the an approximating parallelogram in the tangent plane. If the components $\textbf{r}_u$ and $\textbf{r}_v$ are nonzero and nonparallel in $D$, then
\begin{equation}\label{eq:SurfaceIntegral}
\iint_S{f(x,y,z)\,dS} = \iint_D{f(\textbf{r}(u,v))|\textbf{r}_u \times \textbf{r}_v|\,dA}
\end{equation}
where 
\begin{equation*}
\textbf{r}_u = \frac{\partial x}{\partial u}\,\textbf{i} + \frac{\partial y}{\partial u}\,\textbf{j} + \frac{\partial z}{\partial u}\,\textbf{k}\qquad\textbf{r}_v = \frac{\partial x}{\partial v}\,\textbf{i} + \frac{\partial v}{\partial v}\,\textbf{j} + \frac{\partial z}{\partial v}\,\textbf{k}
\end{equation*}
which is comparable to the formula for a line integral:
\begin{equation*}
\int_C{f(x,y,z)\,ds} = \int_a^b{f(\textbf{r}(t))|\textbf{r}'(t)|\,dt}
\end{equation*}
Formula (\ref{eq:SurfaceIntegral}) allows to compute the surface integral of $S$ be converting into a double integral over the parameter domain $D$.

\subsection{Graphs}
Let surface $S$ be represented by the equation $z = g(x,y)$, then equation (\ref{eq:SurfaceIntegral}) can be used with the following parameterization:
\begin{equation*}
x = x(u,v) \qquad y = y(u,v) \qquad z = g(u,v)
\end{equation*}
which yields
\begin{equation*}
|\textbf{r}_x \times \textbf{r}_y| = \sqrt{\Bigg(\frac{\partial z}{\partial x}\Bigg)^2 + \Bigg(\frac{\partial z}{\partial y}\Bigg)^2 + 1}
\end{equation*}
Therefore, equation (\ref{eq:SurfaceIntegral} becomes
\begin{equation*}
\iint_S{f(x,y,z)\,dS} = \iint_D{f(x,y,g(x,y))\sqrt{\Bigg(\frac{\partial z}{\partial x}\Bigg)^2 + \Bigg(\frac{\partial z}{\partial y}\Bigg)^2 + 1}}
\end{equation*}
Similar formulas exist if it is convenient to project $S$ onto the $yz$-plane or $xz$-plane.

\end{document}
