\documentclass[../main.tex]{subfiles}

\begin{document}
Let $C$ be a smooth curve (continuous and non-zero derivative) given by the parametric equations
\begin{equation*}
\textbf{r}(t) = x(t)\,\textbf{i} + y(t)\,\textbf{j}
\end{equation*}
Dividing $C$ into $n$ subarcs with lengths $\Delta s_i$, choosing any point $P_i(x_i,y_i)$ on the $i$th subarc, evaluating $f(x_i,y_i)$ and multiplying by the length of $\Delta s_i$ yields
\begin{equation*}
\sum_{i=1}^n{f(x_i,y_i)\Delta s_i}
\end{equation*}
Imagine a finite sequence of rectangles positioned on $C$ with height $f(x_i,y_i)$ and length $\Delta s_i$.

%%%%%%%%%%%%%%%%%%%%%%%%%%%%%%%%%%%%%%%%%%%%%%%%%%
% Definition: Line Integral in Plane
%%%%%%%%%%%%%%%%%%%%%%%%%%%%%%%%%%%%%%%%%%%%%%%%%%
\begin{definition}\label{def:LineIntegral}
If $f$ is defined on a smooth curve $C$ given by $\textbf{r}(t) = x(t)\,\textbf{i} + y(t)\,\textbf{j}$, then the \textbf{line integral of $f$ along $C$} is
\begin{equation*}
	\int_C{f(x,y)\,ds = \lim_{n\rightarrow\infty}{\sum_{i=1}^n{f(x_i,y_i)\Delta s_i}}}
\end{equation*}
\end{definition}

Since the length of $C$ is
\begin{equation*}
	L = \int_a^b{\sqrt{\frac{dx}{dt}^2+\frac{dy}{dt}^2} dt}
\end{equation*}
and if $f$ is continuous, we have
\begin{equation*}
\int_C{f(x,y)\,ds} = \int_a^b{f\big(x(t),y(t)\big)\sqrt{\frac{dx}{dt}^2+\frac{dy}{dt}^2} dt}
\end{equation*}
The value of the line integral does not depend on the parameterization of the curve provided that the curve is traversed exactly once as $t$ increases from $a$ to $b$.

\color{red} Definitions for the line integral with respect to $x$ or $y$ exist, but what is the significance of these definitions? How to visualize...
\color{black}

A vector representation of the line segment that starts at $\textbf{r}_0$ and ends at $\textbf{r}_1$ is given by
\begin{equation*}
\textbf{r}(t) = (1-t)\textbf{r}_0+t\textbf{r}_1\qquad 0\le t\le 1
\end{equation*}

\subsection{Line Integrals in Space}
Similar to definition \ref{def:LineIntegral} (line integrals in a plane) the line integral of $C$ a smooth curve in space given by the vector equation $\textbf{r}(t) = x(t)\,\textbf{i} + y(t)\,\textbf{j} + z(t)\,\textbf{k}$ where $a\le t\le b$ with respect to length is
\begin{equation*}
	\int_C{f(x,y,z)\,ds} = \int_a^b{f(\textbf{r}(t))|\textbf{r}'(t)|}
\end{equation*}

\subsection{Line Integrals of Vector Fields}
%%%%%%%%%%%%%%%%%%%%%%%%%%%%%%%%%%%%%%%%%%%%%%%%%%
% Definition: Line Integral of Vector Field
%%%%%%%%%%%%%%%%%%%%%%%%%%%%%%%%%%%%%%%%%%%%%%%%%%
\begin{definition}
Let $\textbf{F}$ be a continuous vector field defined on a smooth curve $C$ given by a vector function $\textbf{r}(t), a\le t \le b$. Then the \textit{line integral of $F$ along $C$} is
\begin{equation*}
\int_C{\textbf{F}\cdot d\textbf{r}} = \int_a^b{\textbf{F}(\textbf{r}(t))\cdot \textbf{r}'(t) dt = \int_C{\textbf{F}\cdot\textbf{T}ds}}
\end{equation*}
\end{definition}
Integrals with respect to arc length are independent of orientation.

\end{document}
