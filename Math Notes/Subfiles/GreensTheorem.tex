\documentclass[../main.tex]{subfiles}

\begin{document}
\textbf{Green's Theorem} gives the relationship between a line integral around a simple closed curve $C$ and a double integral over the plane region $D$ bounded by $C$. By convention, \textbf{positive orientation} of a simple closed curve $C$ refers to a single \textit{counterclockwise} traversal of $C$.

%%%%%%%%%%%%%%%%%%%%%%%%%%%%%%%%%%%%%%%%%%%%%%%%%%
% Theorem: Green's Theorem
%%%%%%%%%%%%%%%%%%%%%%%%%%%%%%%%%%%%%%%%%%%%%%%%%%
\begin{theorem}[Green's Theorem]
Let $C$ be a positively oriented, piecewise-smooth, simple closed curve in the plane and let $D$ be the region bounded by $C$. If $P$ and $Q$ have continuous partial derivatives on an open region that contains $D$, then
\begin{equation*}
	\int_C{P\ dx + Q\ dy} = \iint_D{\Big(\frac{\partial Q}{\partial x} - \frac{\partial P}{\partial y }\Big)dA}
\end{equation*}
in vector form,
\begin{equation}
	\oint_C{\textbf{F}\cdot d\textbf{r}} = \iint_D{(\text{curl}\ \textbf{F})\cdot \textbf{k}\ dA}
\end{equation}
which expresses the line integral of the tangential component of $\textbf{F}$ along $C$ as the double integral of the vertical component of curl $\textbf{F}$ over the region $D$ enclosed by $C$ and in the other vector form
\begin{equation}
	\oint_C{\textbf{F}\cdot \textbf{n}\ ds} = \iint_D{\text{div}\,\textbf{F}(x,y)\ dA}
\end{equation}
which expresses the line integral of the normal component of $\textbf{F}$ along $C$ as the double integral of the divergence of $\textbf{F}$ over the region $D$ enclosed by $C$.

\end{theorem}
Note the similarities between Green's Theorem and the fundamental theorem of calculus
\begin{equation*}
\int_a^b{F'(x)\ dx} = F(b) - F(a)
\end{equation*}
\color{red}Why do we differentiate between types of regions (i.e. Type I, Type II, and Type III)? Isn't each type simply an integration over an area projected onto either the $xy$-plane, $xz$-plane, or $yz$-plane?\color{black}
\begin{proof}
Green's Theorem is true if and only if
\begin{equation}\label{eq:Green1}
\int_C{P\ dx} = -\iint_D{\frac{\partial P}{\partial y} dA}
\end{equation}
and
\begin{equation}\label{eq:Green2}
\int_C{Q\ dy} = -\iint_D{\frac{\partial Q}{\partial x} dA}
\end{equation}
Let $D$ be a type I region and consider equation (\ref{eq:Green1}):
\begin{equation*}
D = \{(x,y): a \le x \le b \text{ and } g_1(x) \le y \le g_2(x)\}
\end{equation*}
\begin{align*}
-\int_a^b{\int_{g_1(x)}^{g_2(x)}{\frac{\partial P}{\partial y}\ dy\ dx}} &= -\int_a^b{P(g_2(x))-P(g_1(x))\ dx} 
\end{align*}
Let $C = C_1 \cup C_2 \cup C_3 \cup C_4$ and the (LHS) of equation (\ref{eq:Green1}) will achieve the same result. Showing equation (\ref{eq:Green2}) uses similar steps.
\end{proof}
\subsection{Applications of Green's Theorem}
\begin{itemize}
	\item Calculating $\iint_D{\frac{\partial Q}{\partial y} - \frac{\partial P}{\partial x}\ dA}$ may be easier then $\int_C{P\ dx + Q\ dy}$
	\item When calculating areas (i.e. $\iint_D{1\ dA}$), we can choose any $P$ and $Q$ such that $\frac{\partial Q}{\partial x} - \frac{\partial P}{\partial y} = 1$ which can yield the following equality:
	\begin{equation*}
	A = \oint_C{x\ dy} = -\oint_C{y\ dx} = \frac{1}{2}\oint_C{x\ dy - y\ dx}
	\end{equation*}
\end{itemize}
\end{document}
