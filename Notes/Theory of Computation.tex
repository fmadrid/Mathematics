\documentclass{article}
\usepackage[utf8]{inputenc}
\usepackage[english]{babel}

\usepackage{color}			% \color
\usepackage{amsmath}
\usepackage{amsfonts}
\usepackage{amssymb}
\usepackage{amsthm}

\usepackage{upgreek}		% \upvarphi
\usepackage{graphicx}		% \includegraphics
\usepackage{float}			% Figure Placement

\usepackage{subfiles}

\theoremstyle{definition}
\newtheorem{definition}{Definition}
\newtheorem{theorem}{Theorem}
\newtheorem{corollary}{Corollary}
\newtheorem{claim}{Claim}

\begin{document}
\title{Theory of Computation Notes}
\author{Frank Madrid}
\maketitle
\tableofcontents

\section{Reducibility}
A \textbf{reduction} is a way of converting one problem to another problem in such a way that a solution to the second problem can be used to solve the first problem. Given two problems $A$ and $B$, if $A$ reduces to $B$, we can use a solution to $B$ to solve $A$.

\subsection{Mapping Reducibility}
\begin{definition}
	Lanague $A$ is \textbf{mapping reducible} to language$B$, written $A\le_m B$, if there is a computable function $f:\Sigma^* \rightarrow \Sigma^*$, where for every $w$,\begin{equation*}
	w\in A \leftrightarrow f(w) \in B.
	\end{equation*}
	The function $f$ is called the \textbf{reduction} from $A$ to $B$.
\end{definition}
\end{document}