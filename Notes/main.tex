\documentclass{article}
\usepackage[utf8]{inputenc}
\usepackage[english]{babel}

\usepackage{color}			% \color
\usepackage{amsmath}
\usepackage{amsfonts}
\usepackage{amssymb}
\usepackage{amsthm}

\usepackage{upgreek}		% \upvarphi
\usepackage{graphicx}		% \includegraphics
\usepackage{float}			% Figure Placement

\usepackage{subfiles}

\theoremstyle{definition}
\newtheorem{definition}{Definition}
\newtheorem{theorem}{Theorem}
\newtheorem{corollary}{Corollary}
\newtheorem{claim}{Claim}

\begin{document}
\title{Math Notes}
\maketitle
\tableofcontents
\newpage

\section{Matrices}
\subfile{Subfiles/Matrices}

\section{Finite Poisson}
\subfile{Subfiles/FinitePoisson}

\section{Conjugate Gradient}
\subfile{Subfiles/ConjugateGradient}

\section{Eigenvalue Decomposition}
\subfile{Subfiles/EigenvalueDecomposition}

\section{Orthogonal Matrices}
\subfile{Subfiles/OrthogonalMatrices}

\section{Vector Fields}
\subfile{Subfiles/VectorFields}

\section{Line Integrals}
\subfile{Subfiles/LineIntegrals
}
\section{Fundamental Theorem for Line Integrals}
\subfile{Subfiles/FundamentalTheoremForLineIntegrals}

\section{Curl}
\subfile{Subfiles/Curl}

\section{Divergence}
\subfile{Subfiles/Divergence}

\section{Green's Theorem}
\subfile{Subfiles/GreensTheorem}

\section{Surface Integrals}
\subfile{Subfiles/SurfaceIntegrals}

\section{Stoke's Theorem}
\subfile{Subfiles/StokesTheorem}

\section{Derivative}
\subfile{Subfiles/Derivatives}
\end{document}
