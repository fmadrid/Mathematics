\documentclass[../main.tex]{subfiles}

\begin{document}
The divergence represents the net rate of change at a point.
\begin{definition}[Divergence]
If $\textbf{F} = P\,\textbf{i} + Q\,\textbf{j} + R\,\textbf{k}$ is a vector field on $\mathbb{R}^3$ and $\nabla \textbf{F}$ exists, then the \textbf{divergence} of $\textbf{F}$ is the function of three variables defined by
\begin{equation}
\text{div}\ \textbf{F} = \frac{\partial P}{\partial x} + \frac{\partial Q}{\partial y} + \frac{\partial R}{\partial z}
\end{equation}
\end{definition}
\noindent This definition is easy to remember as
\begin{equation}
\text{div}\ \textbf{F} = \nabla \cdot \textbf{F}
\end{equation}
\begin{theorem}[\color{red}No flow through curl?\color{black}]
If $\textbf{F} = P\,\textbf{i} + Q\,\textbf{j} + R\,\textbf{k}$ is a vector field on $\mathbb{R}^3$ and $P$, $Q$, and $T$ have continuous second-order partial derivatives, then
\begin{equation}
\text{div curl}\ \textbf{F} = 0
\end{equation}
\end{theorem}
Note the scalar triple analogy $a \cdot (a \times b ) = 0$.
\end{document}
