\documentclass[../main.tex]{subfiles}

\begin{document}
\textbf{Curl} is an operation on a vector field that represents the rotation at a point $(x,y,z)$ about the axis that points in the direction of curl$(\textbf{F}(x,y,z))$.

%%%%%%%%%%%%%%%%%%%%%%%%%%%%%%%%%%%%%%%%%%%%%%%%%%
% Definition: Curl
%%%%%%%%%%%%%%%%%%%%%%%%%%%%%%%%%%%%%%%%%%%%%%%%%%
\begin{definition}[Curl]
If $\textbf{F} = P\,\textbf{i} + Q\,\textbf{j} + R\,\textbf{k}$ is a vector field on $\mathbb{R}^3$ and the partial derives of $P$, $Q$, and $R$ all exist, then the \textbf{curl} of $\textbf{F}$ is the vector field on $\mathbb{R}^3$ defined by
\begin{equation*}
\text{curl}\ \textbf{F} = \Big( \frac{\partial R}{\partial y} - \frac{\partial Q}{\partial z}\Big)\textbf{i}+\Big( \frac{\partial P}{\partial z} - \frac{\partial R}{\partial x}\Big)\textbf{j}+\Big( \frac{\partial Q}{\partial x} - \frac{\partial P}{\partial y}\Big)\textbf{k}
\end{equation*}
\color{red} Looks like a vector component version of Green's Theorem. The Latex command for $\nabla$ is \textit{nabla} but it is called \textit{del}?\color{black}
\end{definition}
This definition is easier to remember as
\begin{equation*}
\text{curl}\ \textbf{F} = \nabla \times \textbf{F}
\end{equation*}
\begin{theorem}[$\textbf{F}$ conservative $\rightarrow$ curl is $0$]
If $f$ is a function of three variables that has continuous second-order partial derivatives, then
\begin{equation}
\text{curl}(\nabla f) = 0
\end{equation}
\end{theorem}
\noindent$\textbf{F}$ conservative implies $\textbf{F}=\nabla f$, then $\textbf{F}$ conservative implies curl$(\textbf{F}) = 0$.
\begin{theorem}[Restrictive converse]
If $\textbf{F}$ is a vector field defined on all of $\mathbb{R}^3$ whose component functions have continuous partial derivatives and curl $\textbf{F} = 0$, then $\textbf{F}$ is a conservative vector field.
\end{theorem}
\end{document}
