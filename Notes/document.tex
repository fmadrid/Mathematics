\documentclass{article}
\usepackage[utf8]{inputenc}
\usepackage[english]{babel}

\usepackage{color}			% \color
\usepackage{amsmath}
\usepackage{amsfonts}
\usepackage{amssymb}
\usepackage{amsthm}

\usepackage{upgreek}		% \upvarphi
\usepackage{graphicx}		% \includegraphics
\usepackage{float}			% Figure Placement

\usepackage{subfiles}

\theoremstyle{definition}
\newtheorem{definition}{Definition}
\newtheorem{theorem}{Theorem}
\newtheorem{corollary}{Corollary}
\newtheorem{claim}{Claim}

\begin{document}
\title{A Probabilistic Theory of Pattern Recognition}
\maketitle
\tableofcontents

\section{Introduction}
\begin{definition}
	The \textbf{probability of error} for a classifier $g$ is
	\begin{equation}
	L(g)=\mathbf{P}\{g(X)\ne Y\}
	\end{equation}
\end{definition}
\begin{definition}
	The best possible classifier, $g^*$ is defined by
	\begin{equation}
	g^*=\text{argmin}_{g:\mathcal{R}^d\rightarrow\{1,\ldots,M\}}{\mathbf{P}\{g(x)\ne Y\}}
	\end{equation}
\end{definition}
\begin{definition}
	The \textbf{Bayes error} is the minimal probability of error and is denoted by $L* = L(G^*)$
\end{definition}
\begin{definition}
	A classifier is good if it is \textbf{consistent}, that is
	\begin{equation}
		\lim_{n\rightarrow \infty}{\mathbf{E}L_n=L^*}
	\end{equation}
\end{definition}





\end{document}